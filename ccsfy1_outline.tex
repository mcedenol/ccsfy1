\documentclass[11pt,]{article}
\usepackage[left=1in,top=1in,right=1in,bottom=1in]{geometry}
\newcommand*{\authorfont}{\fontfamily{phv}\selectfont}
\usepackage[]{mathpazo}


  \usepackage[T1]{fontenc}
  \usepackage[utf8]{inputenc}



\usepackage{abstract}
\renewcommand{\abstractname}{}    % clear the title
\renewcommand{\absnamepos}{empty} % originally center

\renewenvironment{abstract}
 {{%
    \setlength{\leftmargin}{0mm}
    \setlength{\rightmargin}{\leftmargin}%
  }%
  \relax}
 {\endlist}

\makeatletter
\def\@maketitle{%
  \newpage
%  \null
%  \vskip 2em%
%  \begin{center}%
  \let \footnote \thanks
    {\fontsize{18}{20}\selectfont\raggedright  \setlength{\parindent}{0pt} \@title \par}%
}
%\fi
\makeatother




\setcounter{secnumdepth}{0}


\usepackage{graphicx}
% We will generate all images so they have a width \maxwidth. This means
% that they will get their normal width if they fit onto the page, but
% are scaled down if they would overflow the margins.
\makeatletter
\def\maxwidth{\ifdim\Gin@nat@width>\linewidth\linewidth
\else\Gin@nat@width\fi}
\makeatother
\let\Oldincludegraphics\includegraphics
\renewcommand{\includegraphics}[1]{\Oldincludegraphics[width=\maxwidth]{#1}}

\title{Health Effects of Extreme Heat Among Senior Residents of Affordable
Housing \thanks{Special thanks to Alex Hem and Anna-Kate Hard for their assistance
during data collection. \textbf{Current version}: July 05, 2017;
\textbf{Corresponding author}:
\href{mailto:memocedeno@mail.harvard.edu}{\nolinkurl{memocedeno@mail.harvard.edu}}}  }



\author{\Large Augusta Williams\vspace{0.05in} \newline\normalsize\emph{Harvard T.H. Chan School of Public Health}   \and \Large Jose Guillermo Cedeno Laurent, ScD\vspace{0.05in} \newline\normalsize\emph{Harvard T.H. Chan School of Public Health}  }


\date{}

\usepackage{titlesec}

\titleformat*{\section}{\normalsize\bfseries}
\titleformat*{\subsection}{\normalsize\itshape}
\titleformat*{\subsubsection}{\normalsize\itshape}
\titleformat*{\paragraph}{\normalsize\itshape}
\titleformat*{\subparagraph}{\normalsize\itshape}


\usepackage{natbib}
\bibliographystyle{apsr}
\usepackage[strings]{underscore} % protect underscores in most circumstances



\newtheorem{hypothesis}{Hypothesis}
\usepackage{setspace}

\makeatletter
\@ifpackageloaded{hyperref}{}{%
\ifxetex
  \PassOptionsToPackage{hyphens}{url}\usepackage[setpagesize=false, % page size defined by xetex
              unicode=false, % unicode breaks when used with xetex
              xetex]{hyperref}
\else
  \PassOptionsToPackage{hyphens}{url}\usepackage[unicode=true]{hyperref}
\fi
}

\@ifpackageloaded{color}{
    \PassOptionsToPackage{usenames,dvipsnames}{color}
}{%
    \usepackage[usenames,dvipsnames]{color}
}
\makeatother
\hypersetup{breaklinks=true,
            bookmarks=true,
            pdfauthor={Augusta Williams (Harvard T.H. Chan School of Public Health) and Jose Guillermo Cedeno Laurent, ScD (Harvard T.H. Chan School of Public Health)},
             pdfkeywords = {extreme heat, indoor temperature, vulnerable populations, health effects},  
            pdftitle={Health Effects of Extreme Heat Among Senior Residents of Affordable
Housing},
            colorlinks=true,
            citecolor=blue,
            urlcolor=blue,
            linkcolor=magenta,
            pdfborder={0 0 0}}
\urlstyle{same}  % don't use monospace font for urls



% add tightlist ----------
\providecommand{\tightlist}{%
\setlength{\itemsep}{0pt}\setlength{\parskip}{0pt}}

\begin{document}
	
% \pagenumbering{arabic}% resets `page` counter to 1 
%
% \maketitle

{% \usefont{T1}{pnc}{m}{n}
\setlength{\parindent}{0pt}
\thispagestyle{plain}
{\fontsize{18}{20}\selectfont\raggedright 
\maketitle  % title \par  

}

{
   \vskip 13.5pt\relax \normalsize\fontsize{11}{12} 
\textbf{\authorfont Augusta Williams} \hskip 15pt \emph{\small Harvard T.H. Chan School of Public Health}   \par \textbf{\authorfont Jose Guillermo Cedeno Laurent, ScD} \hskip 15pt \emph{\small Harvard T.H. Chan School of Public Health}   

}

}







\begin{abstract}

    \hbox{\vrule height .2pt width 39.14pc}

    \vskip 8.5pt % \small 

\noindent Extreme heat events are increasing in frequency, severity and intensity
due to climate change. Exposure to high temperatures might be
exacerbated by the high thermal inertia of buildings constructed with
retention of heat gains in mind. We followed senior residents of
affordable housing living in two types of buildings (window AC, n=XX;
central AC, n=YY) before and during and extreme heat event. Occupants of
non-AC units report an increase of YY\% in heat related health symptoms,
as well as other symptom groups (i.e., mental health, respiratory)
compared to AC occupants. Moreover, we found that a differential
reporting of symptoms based on existing preconditions, such as YY1, YY2,
YY3. A closer look into the hourly indoor temperature profiles, units
where AC was used earlier after the onset of extreme heat decrease
thermal exposures in XX C/day. This exposure reduction reflects in the
prevalence of different health symptom groups. Building characteristics
modify the health condition of seniors during extreme heat events. Given
the importance of indoor heat exposures, we propose the creation of a
metric that quantifies the risk associated to their residential space
and existing preconditions


\vskip 8.5pt \noindent \emph{Keywords}: extreme heat, indoor temperature, vulnerable populations, health effects \par

    \hbox{\vrule height .2pt width 39.14pc}



\end{abstract}


\vskip 6.5pt

\noindent  \section{Introduction}\label{introduction}

\begin{enumerate}
\def\labelenumi{\arabic{enumi})}
\tightlist
\item
  Talk about the increasing number of heat events in the last 10 years
  or so, and stress the point that heat is the greatest contributor to
  mortality associated to natural disasters.
\item
  Few field studies monitoring indoor temperatures and behavior.
\item
  Increased thermal mass prolonguing heat exposures beyond official
  heatwave duration.
\end{enumerate}

\section{Methods}\label{methods}

\subsection{Baseline and daily
surveys}\label{baseline-and-daily-surveys}

Baseline administered by recruitment team and self-administered daily
surveys. Type of questions, (e.g, time-activity log, symptoms, sleep
quality,etc).

\subsection{Environmental sampling}\label{environmental-sampling}

Netamos, parameters, logging rate. Criteria for installation, etc.

\subsection{Personal monitoring}\label{personal-monitoring}

Basis watches

\section{Results}\label{results}

Start placing demographic table and indoor env. quality parameters here

\subsubsection{Table 1. Findings from someone
else}\label{table-1.-findings-from-someone-else}

\begin{table}[!htbp] \centering 
  \caption{A Handsome Table} 
  \label{} 
\begin{tabular}{@{\extracolsep{5pt}}lc} 
\\[-1.8ex]\hline 
\hline \\[-1.8ex] 
 & \multicolumn{1}{c}{\textit{Dependent variable:}} \\ 
\cline{2-2} 
\\[-1.8ex] & vote \\ 
\hline \\[-1.8ex] 
 z.age & 0.575$^{***}$ \\ 
  & (0.151) \\ 
  & \\ 
 female & 0.310$^{**}$ \\ 
  & (0.151) \\ 
  & \\ 
 z.education & 0.459$^{**}$ \\ 
  & (0.184) \\ 
  & \\ 
 z.income & 0.739$^{***}$ \\ 
  & (0.170) \\ 
  & \\ 
 Constant & 1.706$^{***}$ \\ 
  & (0.110) \\ 
  & \\ 
\hline \\[-1.8ex] 
Observations & 1,500 \\ 
Log Likelihood & $-$592.801 \\ 
Akaike Inf. Crit. & 1,195.602 \\ 
\hline 
\hline \\[-1.8ex] 
\textit{Note:}  & \multicolumn{1}{r}{$^{*}$p$<$0.1; $^{**}$p$<$0.05; $^{***}$p$<$0.01} \\ 
\end{tabular} 
\end{table}

More results

\begin{figure}[htbp]
\centering
\includegraphics{ccsfy1_outline_files/figure-latex/unnamed-chunk-3-1.pdf}
\caption{A Coefficient Plot}
\end{figure}

Notes Adding \texttt{echo="FALSE"} inside the brackets to start the R
chunk will omit the presentation of the R commands. It will just present
the table. This provides substantial opportunity for authors in doing
their analyses. Now, the analysis and presentation in the form of a
polished manuscript can be effectively simultaneous.\footnote{We never
  use footnotes, but just in case we do here is the format.}


\newpage
\singlespacing 
\bibliography{C:/Users/HSPH-HP/Documents/Dropbox/master}

\end{document}
